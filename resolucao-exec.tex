\documentclass[12pt,a4paper, oneside]{book}

\usepackage[top = 2cm, bottom = 2cm, left = 2.5cm, right = 2.5cm]{geometry}
\linespread{1.5}
\usepackage{setspace}
\usepackage[T1]{fontenc}
\usepackage[brazilian]{babel}
\usepackage{color}
\usepackage[dvipsnames,svgnames,x11names]{xcolor}

\usepackage{amsmath,array,amssymb}
\usepackage{subeqnarray}
\usepackage{graphicx,color}
\usepackage{bm}

\begin{document}

%Aqui está como resolução o codigo fonte dos 50 exercicios propostos no capítulo sobre ambiente matemático.

\begin{equation}
\left(\frac{a}{b + c}\right)
\end{equation}

\begin{equation}
(a \times b) + c - \left(\frac{d}{e}\right)
\end{equation}

\begin{equation}
a^x
\end{equation}

\begin{equation}
a^{x + y}
\end{equation}

\begin{equation}
a^{\frac{x}{y}}
\end{equation}

\begin{equation}
a^{\sqrt[b]{c}}
\end{equation}

\begin{equation}
a^{\frac{m}{n}} = \sqrt[m]{n}
\end{equation}

\begin{equation}
\log_3 \, {\sqrt[3]{3}} = x
\end{equation}

\begin{equation}
\frac{1}{\log_x \, 4} + \frac{1}{\log_{2x} \, 4} + \frac{1}{\log_{2x} \, 4} = 1 
\end{equation}

\begin{equation}
f(x) = a\,x^2 + b\,x + c
\end{equation}

\begin{equation}
h(x) = (\sin \, x)^2 + (\cos \, x)^2
\end{equation}

\begin{equation}
\frac{sec\left(2a + \frac{\pi}{3}\right)}{cossec \left(2a + \frac{\pi}{3}\right)} = \frac{\sqrt{3}}{3}
\end{equation}

\begin{equation}
a = -\frac{\pi}{12} + k\frac{\pi}{2}, \; \; k \in \mathbb{Z}
\end{equation}

\begin{equation}
f(x) = 2\,\sin^2\,(4x)  +  \sin\,(6x) - 8
\end{equation}

\begin{equation}
\lim_{x \rightarrow 2} \, \sqrt{\frac{x^3 + 2x + 3}{x^2 + 5}}
\end{equation}

\begin{equation}
\lim_{x \rightarrow -4} \, \frac{x^2 + 5x + 4}{x^2 + 3x -4}
\end{equation}

\begin{equation}
f(x) = \left\lbrace
\begin{array}{cc}
   6x - 1, & x \neq 2 \\
   3, & x = 2
\end{array}
\right.
\end{equation}

\begin{equation}
\int (2\,\sin\,x + 5 \cos x)\, dx
\end{equation}

\begin{equation}
\int (sec^2\,\theta - \sin\,\theta)\,d\theta
\end{equation}

\begin{equation}
\int \left(\frac{2}{\sqrt{1-x^2} - \frac{1}{\sqrt[4]{x}}}\right)\, dx
\end{equation}

\begin{equation}
\int_{0}^{3} f(x)\,dx
\end{equation}

\begin{equation}
\mathcal{A} = \int_{1}^{3}\sqrt{4 - x^2}\,dx
\end{equation}

\begin{equation}
\mathcal{B} = \int_{0}^{2} \vert 2x - 6 \vert \, dx
\end{equation}

\begin{equation}
\int_{-\frac{\pi}{2}}^{\frac{\pi}{2}} \cos\theta\,d\theta
\end{equation}

\begin{equation}
x = \frac{-b \pm \sqrt{b^2 - 4\,a\,c}}{2\,a} \Rightarrow x = \frac{-b \pm \sqrt{\Delta}}{2\,a}
\end{equation}

\begin{equation}
\vec{F} = m\,\vec{a}
\end{equation}

\begin{equation}
\vec{v} = \lambda\,\vec{f}
\end{equation}

\begin{equation}
\vec{\nabla} \cdot \vec{E} = \frac{\rho}{\epsilon_{0}}
\end{equation}

\begin{equation}
\vec{\mathbf{\nabla}} \cdot \vec{\mathbf{B}} = 0
\end{equation}

\begin{equation}
\vec{\nabla} \times \vec{B} = -\frac{\partial \vec{B}}{\partial t}
\end{equation}

\begin{equation}
M_{2x2} = \left(
\begin{array}{lr}
    1 & 2 \\    
    3 & 4 \\
\end{array}
\right)
\end{equation}

\begin{equation}
M_{3x3} = \left(
\begin{array}{lcr}
    1 & 2 & 3 \\    
    4 & 5 & 6 \\
    7 & 8 & 9 \\
\end{array}
\right)
\end{equation}

\begin{equation}
M_{4x4} = \begin{pmatrix}
    1 & 2 & 3 & 4 \\
    5 & 6 & 7 & 8 \\
    9 & 10 & 11 & 12 \\
    13 & 14 & 15 & 16
\end{pmatrix}
\end{equation}

\begin{equation}
M_{2x3} = \begin{pmatrix}
    10 & 5 & 8 \\
    3 & 1 & 4 
\end{pmatrix}
\end{equation}

\begin{equation}
\left(
\begin{array}{lr}
    a & b \\    
    c & d \\
\end{array}
\right) = \left( \begin{array}{lr}
	1 & 2 \\    
    3 & 4 \\
\end{array}
\right)
\end{equation}

\begin{equation}
M_{2x2} = \left(
\begin{array}{lr}
    x+y & t-z \\    
    2x-y & t+z \\
\end{array}
\right)
\end{equation}

\begin{equation}
A = \left(
\begin{array}{lr}
    a & b \\    
    c & d \\
\end{array}
\right) \left(
\begin{array}{lr}
    e & f \\    
    g & h \\
\end{array}
\right)
\end{equation}

\begin{equation}
a_{ij} = \left\lbrace
\begin{array}{cc}
   2^{i+j}, & i < j \\
   i^2 + 5, & i \geq j
\end{array}
\right.
\end{equation}

\begin{eqnarray}
5x + 6y = 10 \\
9x + 15y = 20
\end{eqnarray}

\begin{equation}
M_{4x1} = \left[
\begin{array}{c}
	\frac{1}{2}\\
	\frac{1}{4}\\
	\frac{1}{16}\\
	\frac{1}{32}\\
\end{array}
\right]
\end{equation}

\begin{equation}
X + Y \; \stackrel{5\,min}{\longrightarrow}\; Z + W
\end{equation}

\begin{equation}
A + B \; \stackrel{\Delta}{\longrightarrow}\; C + D
\end{equation}

\begin{equation}
_{94}^{239}Pu \,\longrightarrow\, _{2}^{4}\alpha + \,_{92}^{235}U
\end{equation}

\begin{equation}
N = N_{0} e^{-\lambda t}
\end{equation}

\begin{equation}
\overbrace{XYZ}
\end{equation}

\begin{equation}
\overline{(X \cdot Y)} + \overline{(Z \cdot W)}
\end{equation}

\begin{equation}
\overline{(X \cdot Y) + (Z \cdot W)}
\end{equation}

\begin{equation}
(\overline{X} + Y \cdot W) \cdot (\overline{Z} \cdot \overline{R})
\end{equation}

\begin{equation}
\overrightarrow{ABC} = \overrightarrow{A} \cdot \overrightarrow{B} \cdot \overrightarrow{C}
\end{equation}

\begin{equation}
\det(A) = \left\vert
\begin{array}{ccc}
	1 & 2 & 3 \\
	4 & 5 & 6 \\
	7 & 8 & 9 \\
\end{array}
\right\vert
\end{equation}

\end{document}